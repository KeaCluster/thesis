\chapter{Introducción}
    Se comienza a escribir en este apartado y se cita \cite{kopka1995guide}
	% COMENZAR A ESCRIBIR la MEMORIA
	
	% IMPORTANTE: en LaTeX hay una serie de caracteres especiales, que son: # $ % & \ ^ _ { } ~. Si aparecen en el texto, tendrás un error al compilar. La mayoría se pueden escapar escribiendo \ delante. Para \ utiliza \textbackslash ; para ^ \textasciitilde y para ~ \textasciicircum.

    % COMO incluir una FIGURA siguiendo las recomendaciones de la Guía: Alineación del título: izquierda, en la parte inferior de la figura; fuente a 10pt (el resto de la memoria está a 12); Numeración de la figura: se identificarán con números arábigos consecutivos tras la palabra Tabla. Si se sigue un esquema numérico de capítulos, el número de la tabla debe identificar con su primer dígito al capítulo, seguido por un punto y el número consecutivo que corresponda (Tabla 1.1, etc.); Propiedad intelectual: Se debe indicar la fuente de origen de la información en la parte inferior de la figura, a continuación del título.
    
    % EJEMPLO DE INCLUSIÓN de una figura:
    % \begin{figure}[H]
    % 	\ffigbox[\FBwidth] {
    % 	\caption[Nombre que aparecerá en el índice]{Nombre que aparecerá debajo de la figura}
    % 	}
    % 	{\includegraphics[scale=0.6]{imagenes/creativecommons.png}}
    % \end{figure}
    
    % COMO incluir una TABLA siguiendo las recomendaciones de la Guía: Alineación del título: Centrado en la parte superior de de la tabla, en la parte superior de la tabla; fuente a 10pt (el resto de la memoria está a 12) y en mayúsculas; Numeración: se identificarán con números arábigos consecutivos tras la palabra Tabla. Si se sigue un esquema numérico de capítulos, el número de la tabla debe identificar con su primer dígito al capítulo, seguido por un punto y el número consecutivo que corresponda (Tabla 1.1, etc.); Propiedad intelectual: Se debe indicar la fuente de origen de la información en la parte inferior de la tabla.
    
    % EJEMPLO DE INCLUSIÓN de una tabla:
% \begin{table}[H]
% 	\ttabbox[\FBwidth]
% 	{\caption{Lorem ipsum}}
% 	{\begin{tabular}{|c|P{1.5cm}|c|P{1.5cm}|P{2cm}|c|P{1.5cm}|P{2cm}|}
% 		\hline
% 		\multicolumn{2}{|c|}{\textbf{I}} & \multicolumn{2}{c|}{\textbf{II}} & \multicolumn{3}{c|}{\textbf{III}} & \textbf{IV} \\
% 		\hline
% 		x & y & x & y & x & y & x & y \\
% 		\hline
% 		10.0 & 8.04 & 10.0 & 9.14 & 10.0 & 7.46 & 8.0 & 6.58 \\
% 		\hline
% 		8.0 & 6.95 & 8.0 & 8.14 & 8.0 & 6.77 & 8.0 & 5.76 \\
% 		\hline
% 		13.0 & 7.58 & 13.0 & 8.74 & 13.0 & 12.74 & 8.0 & 7.71 \\
% 		\hline
% 		9.0 & 8.81 & 9.0 & 8.77 & 9.0 & 7.11 & 8.0 & 8.84 \\
% 		\hline
% 		11.0 & 8.33 & 11.0 & 9.26 & 11.0 & 7.81 & 8.0 & 8.47 \\
% 		\hline
% 		14.0 & 9.96 & 14.0 & 8.10 & 14.0 & 8.84 & 8.0 & 7.04 \\
% 		\hline
% 		6.0 & 7.24 & 6.0 & 6.13 & 6.0 & 6.08 & 8.0 & 5.25 \\
% 		\hline
% 		4.0 & 4.26 & 4.0 & 3.10 & 4.0 & 5.39 & 19.0 & 12.50 \\
% 		\hline
% 		12.0 & 10.84 & 12.0 & 9.13 & 12.0 & 8.15 & 8.0 & 5.56 \\
% 		\hline
% 		7.0 & 4.82 & 7.0 & 7.26 & 7.0 & 6.42 & 8.0 & 7.91 \\
% 		\hline
% 		5.0 & 5.68 & 5.0 & 4.74 & 5.0 & 5.73 & 8.0 & 6.89 \\
% 		\hline
% 		\multicolumn{5}{l}{Fuente: BOE}
% 	\end{tabular}}
% \end{table}